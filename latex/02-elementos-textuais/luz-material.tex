\chapter{A LUZ COMO MATERIAL E O CUBO PRETO}

Segundo \citeonline[p. 1]{azevedo} a problemática da luz atravessa a história da arte, de finais do século XIX e durante o século XX. A função da luz não é mais somente de iluminar, de tornar visível uma obra ou um objeto, ou o mero reflexo dos seus efeitos suspensos no espaço. A luz passa a ser tratada como objeto ou como material. Na perspectiva da arte contemporânea, se vê que, em muitas obras, a luz passa à matéria. \citeonline[p. 50]{brandi} destaca que "alguns artistas e movimentos estéticos estão fortemente relacionados com a linguagem da luz, mesmo quando não a utilizam como objeto central da obra". 

De acordo com \citeonline[p. 18]{vega} os artistas, em diferentes épocas, se viram inspirados e cativados pela luz, tanto a natural quanto a artificial, e tentaram capturar seu mistério e sua natureza mágica em suas criações. Alguns em particular, como Caravaggio, Vermeer e Monet, buscaram retratar, quase como uma obsessão, a luz e seus efeitos no mundo ao seu redor e, mais recentemente, os artistas contemporâneos se utilizam da luz como matéria-prima, realizando manipulações em três dimensões para projetar suas dimensões infinitas. \citeonline[p. 23]{vega} afirma também que, atualmente, muitos artistas exploram as possibilidades da luz artificial, trabalhando com mescla de materiais e diversos tipos de fontes de luz. 

James Turrell, por exemplo, é um autor que baseia suas investigações artísticas na luz. Em conversa com \citeonline[p. 114]{adcock}, ele relata que uma das dificuldades de usar a luz é que ainda não é tradição usá-la em nossa cultura. Por outro lado, não é mais incomum usá-la do que usar pedra, argila, aço ou tinta. O artista declara seu interessa em usar a luz como material, mas não luz em vidro, fibra de vidro ou acrílico, e sim no próprio espaço e nas qualidades do espaço, fazendo luz sem a forma física tradicional. Ele nos traz também que há uma rica tradição na pintura do trabalho sobre a luz, mas que de fato não é luz - é o registro da visão; e que, seu material, em contrapartida, não é vicário. Na figura \ref{fig:james_turrell} podemos ver o seu trabalho entitulado \textit{The light inside} que transforma as paredes de um túnel em vasos para a condução da luz. 

\begin{figure}[H]
    \centering
    \caption{The light inside, James Turrell, 1999}
	\vspace*{0,2cm}
    \includegraphics[width=0.8\textwidth]{./04-figuras/james_turrell}
    \label{fig:james_turrell}
\end{figure}
\vspace*{-0,9cm}
{\raggedright \fonte{Disponível em: <http://jamesturrell.com/work/thelightinside/>. Acesso em: 18 jun. 2018}}\\


Outro artista relevante para este trabalho, é o japonês Takahito Matsuo que, segundo \citeonline[p. 5]{soares}, cria mundos interativos de fantasia e de luz que fazem parte de uma estética enigmática (figura \ref{fig:takahito_matsuo}), misturando som e luz perante os movimentos do observador. Seu trabalho destaca as diferentes gradações de luz e sombra que contrastando mostram um mundo de fantasia e imaginação.
\index{Takahito Matsuo}

\begin{figure}[H]
    \centering
    \caption{Fantasias Aquáticas Iluminadas, Takahito Matsuo, 2009}
	\vspace*{0,2cm}
    \includegraphics[width=0.8\textwidth]{./04-figuras/takahito_matsuo}
    \label{fig:takahito_matsuo}
\end{figure}
\vspace*{-0,9cm}
{\raggedright \fonte{\citeonline{soares}}}\\

A figura \ref{fig:jim_campbell} mostra o trabalho do artista Jim Campbell que, em um mundo de alta definição e telas cada vez mais finas, usa tecnologia para produzir o contrário: imagens borradas e em baixa resolução em painéis tridimensionais. Essas video-esculturas são compostas por grades de LEDs que atuam como uma televisão de pixels desconstruída. 

\begin{figure}[H]
    \centering
    \caption{Light Topography Wave, Jim Campbell, 2014}
	\vspace*{0,2cm}
    \includegraphics[width=0.8\textwidth]{./04-figuras/jim_campbell}
    \label{fig:jim_campbell}
\end{figure}
\vspace*{-0,9cm}
{\raggedright \fonte{Disponível em: <https://design-milk.com/pixelated-led-art-jim-campbell/>. Acesso em: 22 mar. 2018}}\\

\citeonline[p. 40]{soares}, afirma que "a maioria dos autores que trabalham com arte e tecnologia procuram o espaço do cubo preto como espaço expositivo. Neste espaço o que interessa é um novo ver, um espanto com a imagem". Diz ainda que o nome cubo preto para este tipo de exposição surge em contraposição ao cubo branco, criado por Brian O'Doherty, num ensaio publicado pela revista Artforum em 1976, fazendo alusão ao espaço das galerias de arte, com paredes brancas, sem janelas isolando o espetador num meio aparentemente atemporal. A ideia do cubo preto surge como ambiente ideal para propagação da luz e é também uma forma de imersão no interior da mente do artista.
