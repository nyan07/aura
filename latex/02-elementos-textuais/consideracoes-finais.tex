\chapter{CONSIDERAÇÕES FINAIS}

O aparecimento de novos materiais não convencionais levanta alguma polemica e encontram adversários, principalmente no meio artístico (AZEVEDO, 2015). Por isso, muito me questionei ao longo de minha trajetória sobre o papel que a tecnologia poderia ocupar nas artes. Inclusive impûs resistência para unir a arte à tecnologia, uma área de conhecimento que já ocupava há mais de uma década. 

No âmbito deste trabalho, a aplicação da tecnologia e a utilização artística da luz, levou-me à construção de objetos que emanam luz e que, conectados, se completam com a presença do observador. São colocados num espaço com um percurso indefinido, mas com área limitada. Dentro desta área o observador é convidado a explorar seus movimentos e percepções. A experiência confirma a luz como material plástico essencial e a tecnologia como meio para sua execução, neste caso, tornando secundário o aspecto da instalação; e requer portanto, uma ação por parte dos observadores, já que o deslocamento dessa plasticidade é construído na sua relação com os mesmos.

O trabalho realizado aqui é uma amostra da potencialidade do uso da tecnologia na arte e serve como ponto de partida para pesquisas futuras mais aprofundadas.


\cite{correia}

\cite{lucero}

\cite{microsoft}

\cite{plaza}

\cite{semeler}

\cite{processing}

\cite{soares}

\citeonline{soares}

