\chapter{INTERAÇÃO E PARTICIPAÇÃO}

De acordo com \citeonline[p. 97]{frechiani}, históricamente, dentre as tendências que situam a importância do espectador na costituição da obra, a arte percorreu um caminho até finalmente desembocar no universo da interatividade. Esse percurso parte da contemplação, onde o fruidor é um agente passivo, passa pela participação ativa, onde encontramos a manipulação de objetos, até finalmente chegar na interação, onde se manifesta uma relação de reciprocidade entre o usuário e um sistema computacional.

Segundo \citeonline{plaza}, foi a partir dos anos cinqüenta que se constituiram, no campo da arte, tendências que traduzem e antecipam as mudanças produzidas pelas tecnologias. O artista se interessa por uma nova forma de comunicação, onde procura a participação do espectador para elaborar a obra de arte, modificando assim o estatuto desta e do autor. A obra deixa de ser fruto apenas do artista e se produz no decorrer do diálogo. O espectador não está mais reduzido ao olhar, ele tem a possibilidade de agir sobre a obra e modificá-la, tornando-se co-autor. Ainda de acordo com \citeonline[p. 14]{plaza} "A noção de arte de participação tem por objetivo encurtar a distância entre criador e espectador. Na participação ativa o espectador se vê induzido à manipulação e exploração do objeto artístico ou de seu espaço". Neste contexto, vivenciamos o início do deslocamento das atribuições do artista para o fruidor. 

\citeonline[p. 22]{domingues} afirma que "a obra interativa pede a participação, a colaboração e só tem existência quando é ativada e modificadas em tempo real dando respostas instantâneas para quem as experimenta". A visão computacional tem um grande papel nessa dinâmica pois são os algoritmos que tornam os trabalhos artísticos mais interativos, imersivos e fazem com que o espectador se constitua como parte da obra. Como visão computacional compreende-se a definição proposta por \citeonline[p. 134]{caetano} que diz "é a forma pela qual 'o computador enxerga' e 'interpreta as imagens', um conjunto de dados numéricos digitais, uma matriz numérica digital descrevendo qualquer conjunto imagético fisicamente contextualizado".

Uma característica marcante da arte interativa é a efemeridade. Em seu texto \textit{A humanização das tecnologias pela arte}, \citeonline[p. 19]{domingues} afirma que a arte que se faz com tecnologias interativas tem como pressupostos básicos a mutabilidade, a conectividade, a não-linearidade, a efemeridade e a colaboração. A arte tecnológica interativa, portanto, pressupõe a parceria, o fim das verdades acabadas, do imutável e do linear. Reforçando essa ideia, \citeonline[p. 24]{venturelli2} nos diz que "a arte que nasce da união artística e a tecnológica é a mais efêmera de todas: a arte do espaço-tempo-movimento. É a arte da ação e do dinamismo". E \citeonline[p. 72]{semeler} propõe que, em projetos de arte e tecnologia, mesmo que a preocupação com o efêmero não apareça como elemento de primeiro plano, ela é decorrente da obsolescência inerente aos dispositivos tecnológicos utilizados, bem como nos efeitos instantâneos produzidos em tempo real pela ação do espectador.

A arte interativa é, portanto, completamente avessa ao principio da inércia. \citeonline[p. 22]{domingues} conclui que surge um novo espectador mais participativo e que, através das interfaces, tem acesso a obra proposta. São as interfaces amigáveis que permitem as trocas do espectador com as fontes de informação. A contemplação é substituída pela relação. Assim, de acordo com \citeonline[p. 20]{plaza} uma obra de arte interativa é um espaço latente e suscetível a todos os prolongamentos sonoros, visuais e textuais. O cenário programado pode se modificar em tempo real ou em função da resposta dos operadores. A interatividade não é somente uma comodidade técnica e funcional; ela implica física, psicológica e sensivelmente o espectador em uma prática de transformação. O destinatário potencial torna-se co-autor e as obras tornam-se um campo aberto a múltiplas possibilidades e suscetível a desenvolvimentos imprevistos numa co-produção de sentidos.		
