\chapter{INTERAÇÃO E PARTICIPAÇÃO}

Segundo \citeonline{plaza}, foi a partir dos anos cinqüenta que se constituiram, no campo da arte, tendências que traduzem e antecipam as mudanças produzidas pelas tecnologias. O artista se interessa por uma nova forma de comunicação, onde procura a participação do espectador para elaborar a obra de arte, modificando assim o estatuto desta e do autor. A obra deixa de ser fruto apenas do artista e se produz no decorrer do diálogo. O espectador não está mais reduzido ao olhar, ele tem a possibilidade de agir sobre a obra e modificá-la, tornando-se co-autor. 

Ainda de acordo com \citeonline[p. 14]{plaza} "A noção de arte de participação tem por objetivo encurtar a distância entre criador e espectador. Na participação ativa o espectador se vê induzido à manipulação e exploração do objeto artístico ou de seu espaço." Já "a arte que se faz com tecnologias interativas tem como pressupostos básicos a mutabilidade, a conectividade, a não-linearidade, a efemeridade, a colaboração. A arte tecnológica interativa pressupõe a parceria, o fim das verdades acabadas, do imutável, do linear." \cite[p.19]{domingues} 

Conforme mencionado acima, nas palavras de \citeonline{domingues}, uma das características da arte interativa é a efemeridade. De acordo com \citeonline[p. 72]{semeler} "apesar de a preocupação com o efêmero não aparecer como elemento de primeiro plano, ela é decorrente da obsolescência inerente aos dispositivos tecnológicos utilizados, bem como nos efeitos instantâneos produzidos em tempo real pela ação do espectador". Reforçando essa ideia, \citeonline[p. 24]{venturelli2} afirma que "a arte que nasce da união artística e a tecnológica é a mais efêmera de todas: a arte do espaço-tempo-movimento. É a arte da ação e do dinamismo".

No meu trabalho ela está representada na luz que apaga ao movimento do espectador e, também, no \textit{hardware} e \textit{software} utilizados que com o passar do tempo se tornam obsoletos. 

Em concordância com essa visão, \citeonline[p.22]{domingues} afirma que "a obra interativa pede a participação, a colaboração e só tem existência quando é ativada e modificadas em tempo real dando respostas instantâneas para quem as experimenta."

\citeonline[p. 20]{plaza} afirma que uma obra de arte interativa é um espaço latente e suscetível a todos os prolongamentos sonoros, visuais e textuais. O cenário programado pode se modificar em tempo real ou em função da resposta dos operadores. A interatividade não é somente uma comodidade técnica e funcional; ela implica física, psicológica e sensivelmente o espectador em uma prática de transformação.  

\citeonline{domingues} concorda com essa visão e, além disso, coloca m pauta o fim da arte de representação em favor de uma arte interativa:

\begin{citacao}
Os artistas ligados a centros avançados de pesquisa ou isoladamente assumem a ruptura com a arte do passado num cenário dominado pela arte da participação, da interação, da comunicação planetária, colocando-se em novos circuitos não mais limitados à arte como objeto ou valor de culto, mas enfatizando, sobretudo, seu poder de comunicação. Fala-se do fim da arte da representação em favor de uma arte interativa que é basicamente comportamental e que não pode se encerrar em objetos acabados como numa escultura, pintura, fotografia ou outro suporte material, nem mesmo no cinema ou no vídeo em seus formatos habituais que impedem o diálogo transformador. \cite[p.17-18]{domingues}\end{citacao}	

\citeonline[p. 20]{plaza} afirma que uma obra de arte interativa é um espaço latente e suscetível a todos os prolongamentos sonoros, visuais e textuais. O cenário programado pode se modificar em tempo real ou em função da resposta dos operadores. A interatividade não é somente uma comodidade técnica e funcional; ela implica física, psicológica e sensivelmente o espectador em uma prática de transformação.  

A visão computacional tem um grande papel nessa dinâmica pois são os algoritmos que tornam os trabalhos artísticos mais interativos, imersivos e fazem com que o espectador se constitua como parte da obra. Como visão computacional compreende-se "a forma pela qual 'o computador enxerga' e 'interpreta as imagens', um conjunto de dados numéricos digitais, uma matriz numérica digital descrevendo qualquer conjunto imagético fisicamente contextualizado" \cite[p. 134]{caetano}.
				
						
A arte interativa é totalmente avessa ao principio da inércia. Surge um novo "espectador" mais participativo que através das interfaces tem acesso a obra proposta. São as interface amigáveis que permitem as trocas do espectador com as fontes de informação. A contemplação é substituída pela relação. \cite[p.22]{domingues}
							
Nas artes da interatividade, portanto, o destinatário potencial torna-se co-autor e as obras tornam-se um campo aberto a múltiplas possibilidades e suscetível a desenvolvimentos imprevistos numa co-produção de sentidos. É assim que nasce a chamada inteligência distribuída ou "coletiva".  \cite[p. 20]{plaza}			
						



