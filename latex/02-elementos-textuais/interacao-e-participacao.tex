\chapter{INTERAÇÃO E PARTICIPAÇÃO}

Segundo Plaza (2009), foi a partir dos anos cinqüenta que se constituiram, no campo da arte, tendências que traduzem e antecipam as mudanças produzidas pelas tecnologias. O artista se interessa por uma nova forma de comunicação, onde procura a participação do espectador para elaborar a obra de arte, modificando assim o estatuto desta e do autor. A obra deixa de ser fruto apenas do artista e se produz no decorrer do diálogo, em tempo real. O espectador não está mais reduzido ao olhar, ele tem a possibilidade de agir sobre a obra e modificá-la, tornando-se co-autor.

A visão computacional tem um grande papel nessa dinâmica pois são os algoritmos que tornam os trabalhos artísticos mais interativos, imersivos e fazem com que o espectador se constitua como parte da obra. Como visão computacional compreende-se "a forma pela qual 'o computador enxerga' e 'interpreta as imagens', um conjunto de dados numéricos digitais, uma matriz numérica digital descrevendo qualquer conjunto imagético fisicamente contextualizado" \cite[p. 134]{caetano}.
				
**Assim, a outra propriedade que permeia os dois desdobramentos decorrentes da arte participativa e interativa é a efemeridade. Na arte que utiliza novas tecnologias, particularmente em minha pesquisa artística, a condição de efêmero é real e determinada pelo hardware e pelo software.**

Segundo Plaza (2009, p. 14) "A noção de "arte de participação" tem por objetivo encurtar a distância entre criador e espectador. Na participação ativa o espectador se vê induzido à manipulação e exploração do objeto artístico ou de seu espaço.”		
							
Nas artes da interatividade, portanto, o destinatário potencial torna-se co-autor e as obras tornam-se um campo aberto a múltiplas possibilidades e suscetível a desenvolvimentos imprevistos numa co-produção de sentidos. É assim que nasce a chamada inteligência distribuída ou "coletiva". (PLAZA, p. 20)				
						
Uma obra de arte interativa é um espaço latente e suscetível a todos os prolongamentos sonoros, visuais e textuais. O cenário programado pode se modificar em tempo real ou em função da resposta dos operadores. A interatividade não é somente uma comodidade técnica e funcional; ela implica física, psicológica e sensivelmente o espectador em uma prática de transformação. (PLAZA, p. 20)


