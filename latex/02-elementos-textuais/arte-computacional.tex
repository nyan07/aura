%
% Documento: Disposições
%

\chapter{ARTE COMPUTACIONAL}

Segundo Azevedo (2005, p. 183) "as novas tecnologias introduzem diferentes problemas de representação, abalam antigas certezas no plano epistemológico e exigem a reformulação de conceitos estéticos". 
De acordo com Semeler (2011) "apesar de a preocupação com o efêmero não aparecer como elemento de primeiro plano, ela é decorrente da obsolescência inerente aos dispositivos tecnológicos utilizados, bem como nos efeitos instantâneos produzidos em tempo real pela ação do espectador". 
								
A utilização de novas tecnologias nas artes visuais leva a um certo modo de entender novas relações formais, implicando um novo modo de perceber, compreender, apreciar as configurações que se apresentam ˜à nossa organização perceptiva. Talvez, entendendo como é que a organização perceptiva se processa, consigamos chegar a um conhecimento consciente da percepção e essa consciência talvez nos possa levar a novas maneiras de percepcionar. (AZEVEDO, 2005)

