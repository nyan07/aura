\chapter{ARTE COMPUTACIONAL}


Segundo \citeonline[p. 183]{azevedo} "as novas tecnologias introduzem diferentes problemas de representação, abalam antigas certezas no plano epistemológico e exigem a reformulação de conceitos estéticos". Assim comoa fotografio foi questionada sobre o seu estatuto de arte, a tecnologia passa pelas mesmas questões. 

De acordo com Semeler (2011) "apesar de a preocupação com o efêmero não aparecer como elemento de primeiro plano, ela é decorrente da obsolescência inerente aos dispositivos tecnológicos utilizados, bem como nos efeitos instantâneos produzidos em tempo real pela ação do espectador". 
								
A utilização de novas tecnologias nas artes visuais leva a um certo modo de entender novas relações formais, implicando um novo modo de perceber, compreender, apreciar as configurações que se apresentam ˜à nossa organização perceptiva. Talvez, entendendo como é que a organização perceptiva se processa, consigamos chegar a um conhecimento consciente da percepção e essa consciência talvez nos possa levar a novas maneiras de percepcionar. (AZEVEDO, 2005)

Para ser considerado um trabalho artístico de arte computacional, ele deve ser projetado para executar processos computacionais – para realizar entradas e saídas de dados de informação, seguindo regras formais, ou algoritmos \cite[p. 132](venturelli)

toda obra de arte computacional contém os seguintes elementos descritivos: 1) é definida como arte pelo meio; 2) é obrigatoriamente executada em um computador; 3) é interativa; e 4) é interativa, porque ela é executada em um computador. Os itens 3 e 4 distinguem obras de arte computacional de trabalhos auxiliados por computador. O que significa isso? Significa que a obra é interativa apenas no caso em que as ações do interagente são prescritas antecipadamente, em parte, gerando concomitantemente a obra, mediada pelo processamento computacional.  \cite[p. 133](venturelli)

Nenhuma obra é interativa a menos que seu software possa variar, dependendo do que seu usuário izer, e isso signi ica que a sua exibição difere de pessoa para pessoa. Uma vez ocorrida a interatividade o usuário passa a apreciar a obra em si. Todos são livres para apreciar a obra como sendo única, não precisando estar ciente de que se realiza uma das muitas facetas possíveis do trabalho. No entanto, para obter a interatividade é fundamental apreciar o trabalho com a utilização de processamento computacional.  \cite[p. 138-139](venturelli)

Assim, os usuários ajudam a gerar e a exibir a obra, sendo papel do artista o de criar alguns itens de possibilidades por meio de variáveis que são, muitas vezes, parte de algum código executado pelo processo computacional, passo a passo. O artista usa seu conhecimento para gerar um trabalho cujo desempenho dependerá do interagente. O computador automatiza a geração do trabalho para que os usuários possam conhecê-lo e explorar as suas muitas exibições. Tanto o usuário quanto o artista fazem parte da obra. O interagente é o artista e se sensibiliza, aprecia o trabalho pela interatividade nele contida, ela é o motivo para a obra computacional ter ou não mérito, segundo alguns críticos oriundos do próprio meio da arte.  \cite[p. 139](venturelli)

