\chapter{ARTE COMPUTACIONAL E INTERATIVIDADE}

\citeonline[p. 36]{boone} afirma que arte computacional é toda arte produzida através de sistemas  computacionais e que, consequentemente, necessita de um computador para a sua existência. \citeonline[p. 132]{venturelli} corrobora com esta ideia ao afirmar que para ser considerado um trabalho artístico de arte computacional, ele deve ser projetado para executar processos computacionais, ou seja, realizar entradas e saídas de dados, seguindo regras formais, ou algoritmos. Em seu artigo \textit{Interatividade computacional}, ela afirma ainda que

\begin{citacao}
toda obra de arte computacional contém os seguintes elementos descritivos: 1) é definida como arte pelo meio; 2) é obrigatoriamente executada em um computador; 3) é interativa; e 4) é interativa, porque ela é executada em um computador. Os itens 3 e 4 distinguem obras de arte computacional de trabalhos auxiliados por computador. O que significa isso? Significa que a obra é interativa apenas no caso em que as ações do interagente são prescritas antecipadamente, em parte, gerando concomitantemente a obra, mediada pelo processamento computacional.  \cite[p. 133]{venturelli}
\end{citacao}

Partindo desta premissa é possível afirmar que toda obra de arte computacional é interativa e que toda obra de arte interativa pode variar de acordo com o que o fruidor fizer. Isso significa que sua exibição difere de pessoa para pessoa. Segundo \citeonline[p. 139]{venturelli} o fruidor ajuda a gerar e exibir a obra, sendo que o papel do artista é o de criar possibilidades através de variáveis que muitas vezes são parte de algum código executado pelo processo computacional. O artista usa seu conhecimento para gerar um trabalho cujo desempenho dependerá do interagente, enquanto o computador automatiza a geração do trabalho para que os espectadores possam conhecer e explorar as suas múltiplas possibilidades de exibição. \citeonline[p. 21]{lister} afirma que em um nível ideológico, a interatividade tem sido uma das principais características da arte computacional. Sendo que, enquanto a arte tradicional oferece consumo passivo, a arte computacional oferecem interatividade.

De acordo com \citeonline{rabello} a interface permite, através de um canal de mão dupla entre homem e máquina, que a ação do homem, por mais sutil e imperceptível que pareça, seja reconhecida, processada pela máquina e devolvida para o interator. Ela traduz as informações de forma que o público compreenda, dando continuidade ao processo interativo no ambiente virtual. As interfaces tornam-se, portanto, um tipo de condutor, estabelecendo a interatividade e convertendo os espectadores em atores dos sistemas. 

\citeonline[p. 11]{bochio} entende que, no contexto digital, a interatividade é a relação recíproca entre usuários e interfaces computacionais. A autora compreende por interfaces os dispositivos de entrada e/ou de saída que funcionam como pontes entre as ações humanas e os códigos do computador. Conclui que é, portanto, uma comunicação baseada na tradução de um código a outro, estabelecendo, assim, um código comum entre homem e máquina. 

Para \citeonline[p. 22]{lister}, no contexto da arte computacional, interatividade se refere a capacidade dos usuários de intervir diretamente e alterar a obra. O público da arte computacional se torna um usuário ao invés de um espectador. É necessário que este usuário intervenha ativamente a fim de produzir significado. Essa intervenção, na verdade, acrescenta outros modos de engajamento, como brincar, experimentar e explorar, sob a idéia de interação. Como afirma \citeonline{campbell}, o artista não escreve o lado do espectador da interação, portanto, o usuário pode responder de maneira mais aberta. Assim, provavelmente, os únicos diálogos significativos que ocorrem durante a interação com um trabalho são entre os espectadores e eles mesmos. As respostas do trabalho são, portanto, reflexos alterados das respostas do espectador e por isso as limitações com as quais nos defrontamos neste momento já não são tecnológicas.