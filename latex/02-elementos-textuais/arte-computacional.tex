\chapter{ARTE COMPUTACIONAL E INTERATIVIDADE}

\citeonline[p.36]{boone} afirma que arte computacional é toda arte produzida através de sistemas  computacionais e que, consequentemente, necessita de um computador para a sua existência. \citeonline[p. 132]{venturelli} corrobora com esta ideia ao afirmar que para ser considerado um trabalho artístico de arte computacional, ele deve ser projetado para executar processos computacionais, ou seja, realizar entradas e saídas de dados, seguindo regras formais, ou algoritmos. Em seu artigo \textit{Interatividade computacional}, ela afirma ainda que

\begin{citacao}
toda obra de arte computacional contém os seguintes elementos descritivos: 1) é definida como arte pelo meio; 2) é obrigatoriamente executada em um computador; 3) é interativa; e 4) é interativa, porque ela é executada em um computador. Os itens 3 e 4 distinguem obras de arte computacional de trabalhos auxiliados por computador. O que significa isso? Significa que a obra é interativa apenas no caso em que as ações do interagente são prescritas antecipadamente, em parte, gerando concomitantemente a obra, mediada pelo processamento computacional.  \cite[p. 133]{venturelli}
\end{citacao}

Partindo desta premissa é possível afirmar que toda obra de arte computacional é interativa e que toda obra de arte interativa pode variar de acordo com o que o fruidor fizer. Isso significa que sua exibição difere de pessoa para pessoa. Segundo \citeonline[p. 139]{venturelli} o fruidor ajuda a gerar e exibir a obra, sendo que o papel do artista é o de criar possibilidades através de variáveis que muitas vezes são parte de algum código executado pelo processo computacional. O artista usa seu conhecimento para gerar um trabalho cujo desempenho dependerá do interagente, enquanto o computador automatiza a geração do trabalho para que os espectadores possam conhecer e explorar as suas múltiplas possibilidades de exibição. \citeonline[p. 21]{lister} diz que em um nível ideológico, a interatividade tem sido uma das principais características da arte computacional. Sendo que, enquanto a arte tradicional oferece consumo passivo, a arte computacional oferecem interatividade.

Para \citeonline[p. 22]{lister}, no contexto da arte computacional, interatividade se refere a capacidade dos usuários de intervir diretamente e alterar a obra. O público da arte computacional se torna um usuário ao invés de um espectador. É necessário que este usuário intervenha ativamente a fim de produzir significado. Essa intervenção, na verdade, acrescenta outros modos de engajamento, como brincar, experimentar e explorar, sob a idéia de interação. 


Um computador pode ser pensado como uma estrutura vazia na qual um conceito é inserido. Este conceito, que deve ser representado de maneira matemática, é o programa. O programa é composto de uma série de algoritmos que definem a resposta do sistema. Entradas acontecem, o programa reage e produz uma saída selecionada do vocabulário daquele sistema particular. Essa saída pode ser uma imagem, um som, um robô que salta para cima e para baixo ou uma mudança na iluminação da sala. São todas formas de representar a direção interna do programa. \cite{campbell}

De que outra forma os artistas podem projetar a interface sem vê-la do outro lado? Uma das maneiras que vi artistas evitarem esse problema é não se colocar no lugar do espectador, mas, em vez disso, ter o ponto de vista do trabalho em si. Em vez de dizer como espectador o que posso desencadear, dizendo como programa o que posso medir? E então, o que posso refletir e o que posso expressar com base em alguma interpretação das respostas do espectador. Desta forma, o trabalho torna-se uma reflexão momentânea, mas dinâmica, de um processo de pensamento. Como o artista não escreve o lado do espectador da interação, o espectador pode responder de maneira mais aberta.

Uma das conseqüências dessa abordagem é que o trabalho como uma pintura e como um filme existe por si só. Não há modo de atração. O trabalho não está esperando por uma pessoa para completá-lo. De certa forma, o trabalho torna-se interativo não com as pessoas, mas com seu ambiente. Isso é particularmente importante no trabalho que existe em um espaço público. O grau em que um trabalho parece um jogo em vez de um diálogo ou o grau em que um trabalho parece uma resposta, em vez de uma pergunta, é a escolha do artista e não uma limitação do meio ou da tecnologia.


