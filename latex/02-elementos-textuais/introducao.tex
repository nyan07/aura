%
% Documento: Introdução
%

\chapter{INTRODUÇÃO}\label{chap:introducao}

\textit{{\imprimirtitulotb}: {\imprimirsubtitulo}} traz à tona a perspectiva da arte computacional, que "envolve sistemas computacionais tanto nos seus processos de criação e produção quanto na forma de apresentação" \cite[p. 36]{boone}, e tem como base a construção de um \textit{software}, bem como a utilização dispositivos eletrônicos, como o computador, Microsoft Kinect e Arduino, para a criação de um ambiente tridimensional e interativo que reage de acordo com a presença do espectador.

Nesta proposta será apresentada uma obra interativa que tem a luz como fonte principal de sua constituição estética. As variações captadas através do sensor Kinect, conectado a um computador, causam o acender e apagar de LEDs (Diodo Emissor de Luz ou \textit{Light Emitting Diode}) dispostos em uma malha presa ao teto. Essa malha é gerenciada por uma placa Arduino. Cada LED, por sua vez, está conectado à fios de fibra ótica que transmitem a luz emitida por eles, gerando um efeito de minúsculos pontos de luz que acompanham o espectador à medida que este caminha pelo espaço.

A presença do espectador, estimulada através do sensor Kinect, provoca o efeito de luzes que acendem e apagam. Neste momento, o fruidor se torna fator indispensável para a constituição da obra. Torna-se obrigatória a manipulação contemplativa dos objetos de luz  (caminhar sob eles) para a existência da mesma. O processo de comunicação entre a instalação e o fruidor, em si, é a obra. O observador faz parte do processo de realização da mesma. O que remete ao efêmero provocado por essa interação e que também está presente no aspecto tecnológico, tanto no \textit{hardware} quanto no \textit{software} utilizados. 

O interesse sobre o tema vem da relação precoce que possuo com sistemas de computação, presente inclusive em minha formação acadêmica. Surge como uma tentativa de fundir dois universos dos quais me sinto parte: a arte e a tecnologia. Além disso, a proposta traz como base de estudo o papel da luz na arte e como esses trabalhos são influenciados pelo meio ao qual são expostos.

A pesquisa engloba imersão na temática, busca de fontes em leituras de textos teóricos, visualização de vídeos, estudos relacionados a eletrônica e a computação. Foram experimentados vários componentes para se obter o resultado proposto. Depois de algumas investidas, chegou-se a composição de LEDs e fibra ótica como elemento luminescente e a utilização do sensor Kinect como entrada de dados devido à sua precisão em relação ao sensor de presença ultrassônico, que foi uma das opções cogitadas.