\chapter{COMPOSIÇÃO DA OBRA}
Esta capítulo está organizado em três partes. Primeiro vamos trazer um breve relato sobre os dispositivos e tecnologias utilizados, depois apresentaremos o protótipo construído para validar o \textit{software} e o \textit{hardware} escolhidos para elaboração da obra e, por fim, vamos fazer uma descrição da solução, apresentando as principais barreiras encontradas ao longo do percurso. 

\section{DISPOSITIVOS E TECNOLOGIAS}

Com o intuito de manter documentada, possibilitando a reprodução ou atualização deste projeto no futuro, se faz necessário listar as versões e configurações aplicadas (quadro \ref{quadro:dispositivos}), bem como introduzir os principais dispositivos e tecnologias utilizados. Dentre os componentes essenciais em termos de \textit{hardware} temos o Microsoft Kinect, o Arduino e o computador; enquanto o principal \textit{software} é a Processing, responsável por orquestrar o funcionamento destes equipamentos em conjunto. 


\begin{quadro}[H]
\caption{\label{quadro:dispositivos}Especificações de \textit{hardware} e \textit{software} utilizados}
\begin{center}  
  \begin{tabulary}{0.8\textwidth}{|L|L|}
  
  \hline
  \textbf{Dispositivo ou \textit{software}} & \textbf{Requisitos} \\ \hline
  Microsoft Kinect  & Versão 1 - Modelo 1414 \\ \hline
  Arduino & Modelo Uno para o protótipo e Modelo Mega 2560 para o trabalho final \\ \hline
  Computador & Suporte a versão da Processing utilizada e disponibilidade de 2 portas USB \\ \hline
  Processing & Versão 3.3.7 \\ \hline
  \end{tabulary}
\end{center}
\vspace*{-0,7cm}
\fonte{Elaborado pela autora}\\
\end{quadro}


A figura \ref{fig:dispositivos} mostra como esses dispositivos estão interligados: o Kinect envia dados de profundidade para o computador, que através de um programa escrito em Processing, realiza cálculos e processamento das informações recebidas e, em seguida, envia instruções para o Arduino. Esse processo é cíclico e constante durante toda a execução da obra.

\begin{figure}[H]
  \begin{center}
    \caption{Dispositivos utilizados na construção da obra}
    \vspace*{0,2cm}
    \includegraphics[width=0.8\textwidth]{./04-figuras/dispositivos}
    \label{fig:dispositivos}
  \end{center}
  \vspace*{-0,9cm}
  \fonte{Elaborada pela autora}\\
\end{figure}


\subsection{Microsoft Kinect}

O sensor Kinect é um dispositivo lançado em 4 de Novembro de 2010 como um acessório do console Xbox 360 da Microsoft. Orientado, principalmente, a indústria de jogos, foi criado para servir como uma forma de interação entre o utilizador e o console Xbox 360 através de gestos e comandos de voz. De acordo com a \citeonline{microsoft}, em sua primeira versão, é capaz de capturar imagens com 640x480 \textit{pixels} a 30 fps. O aparelho é formado por um emissor e sensor de profundidade baseados em infravermelho, uma câmera RGB, um motor de inclinação e uma série de 4 microfones. Na figura \ref{fig:kinect_componentes} podemos ver a posição de cada um destes componentes no dispositivo.

\begin{figure}[H]
  \begin{center}
    \caption{Componentes do sensor Kinect}
    \vspace*{0,2cm}
    \includegraphics[width=0.8\textwidth]{./04-figuras/kinect_componentes}
    \label{fig:kinect_componentes}
  \end{center}
  \vspace*{-0,9cm}
  \fonte{Adaptado de \citeonline{microsoft}}\\
\end{figure}

Dentre seus componentes, o que mais nos interessa no contexto deste trabalho, é o sensor de profundidade. \citeonline{ashley} afirmam que a produção de dados tridimensionais é a principal função do Kinect. Ele difere de qualquer outro dispositivo de entrada justamente porque provê uma terceira dimensão e, para tanto, se utiliza de um emissor e uma câmera de infravermelho. Segundo \citeonline{lucero} o emissor projeta um padrão estruturado de luz infravermelha, enquanto a câmera lê esses raios e interpreta a deformação da projeção, convertendo essa informação em valores de profundidade e, consequentemente, medindo a distância entre o objeto e o sensor. De acordo com \citeonline{correia} estas medidas baseiam-se em triangulação tendo em conta o emissor, a câmera e as posições dos \textit{pixels} no cenário. A profundidade é codificada numa escala de cinzas. Quanto mais escuro o \textit{pixel}, mais próximo do sensor está esse ponto no espaço. Sendo que, \textit{pixels} pretos indicam que não existe informação de profundidade. Isto ocorre no caso dos pontos estarem muito longe, impossibilitando a sua captura, no caso de estarem numa área onde não haja pontos do emissor de infravermelhos, no caso de o objeto refletir mal a luz infravermelha ou, finalmente, no caso de os pontos estarem muito próximos do sensor, uma vez que o campo de visão do Kinect é limitado em cerca de 80 centímetros a 4 metros.

\subsection{Arduino}

O Arduino é uma plataforma de prototipagem eletrônica de \textit{hardware} livre e de placa única \cite{arduino}. De acordo com o \textit{site} oficial do projeto, o objetivo é criar ferramentas acessíveis, com baixo custo, flexíveis e fáceis de usar. Para \citeonline{souza2011} ela é ideal para a criação de dispositivos que permitam interação com o ambiente, podendo utilizar diversas fontes de entrada como, por exemplo, sensores de temperatura, luz ou som, e como saída LEDs, motores, \textit{displays}, auto-falantes, entre outros, criando desta forma possibilidades ilimitadas. É possível dizer à placa o que fazer enviando uma série de instruções ao microcontrolador. Para isso é necessário utilizar a linguagem de programação do Arduino (baseada em Wiring) e o seu \textit{software} (IDE - \textit{Integrated Development Environment} ou Ambiente de Desenvolvimento Integrado), baseada em Processing \cite{arduino}. 


\subsection{Processing}
\label{sec:processing}

De acordo com \citeonline[p. 115]{santos} Processing é a primeira ferramenta criada para artistas por artistas e o seu desenvolvimento foi iniciado no MIT Media Lab por dois estudantes de graduação: Casey Reas e Benjamin Fry. Segundo informações contidas no \textit{site} oficial do projeto, \citeonline{processing} é uma plataforma e uma linguagem de programação de código aberto (\textit{open source}) para prototipação de \textit{software} dentro do contexto das artes visuais. Disponível desde 2001, a Processing vem promovendo a alfabetização em \textit{software} dentro das artes visuais e a alfabetização visual dentro da tecnologia. 

\subsection{Computador}
\label{sec:computador}

Este trabalho requer a utilização de um computador para processar os dados capturados pelo Kinect e enviar ao Arduino. O projeto foi desenvolvido e testado em equipamentos com configurações distintas, sendo que seus únicos requisitos são o suporte à instalação da versão de Processing mencionada no quadro \ref{quadro:dispositivos}, disponível no início deste capítulo, e duas portas USB, uma para cada dispositivo.



\section{PROTOTIPAÇÃO E TESTES}

Para validar o projeto foi construído um protótipo com o Microsoft Kinect, um computador, um Arduino Uno e 5 LEDs conectados à ele conforme pode ser visto na figura \ref{fig:prototipo}. Através da utilização de duas bibliotecas, \textit{Open Kinect for Processing} e Firmata, construiu-se um \textit{script} simplificado que controlava os dispositivos de entrada e saída de forma integrada, causando, assim, o acender e apagar dos LEDs de acordo com as informações capturadas pelo sensor. Desejava-se provar a possibilidade de, primeiro, controlar os dispositivos de entrada e saída de dados de maneira simultânea e, depois, a capacidade da luz emitida pelo LED se propagar através da fibra ótica.

\begin{figure}[H]
  \begin{center}
    \caption{Componentes do prótotipo}
    \vspace*{0,2cm}
    \includegraphics[width=0.8\textwidth]{./04-figuras/prototipo}
    \label{fig:prototipo}
  \end{center}
  \vspace*{-0,9cm}
  \fonte{Elaborada pela autora}\\
\end{figure}

No que diz respeito a integração dos elementos presentes na obra, foi possível constatar seu correto funcionamento já em conformidade com o modelo da proposta introduzida na \fullref{sec:solucao}, pois os dispositivos utilizados para prototipação são os mesmos ou muito similares aos que encontramos no trabalho final. A partir disso, o desafio se concentrou em construir a malha de LEDs, bem como fazer a colagem da fibra ótica em cada um deles, além de criar a versão final do \textit{script} que deveria funcionar como uma matriz e carecia de otimização em sua lógica de detecção de presença.

Na figura \ref{fig:prototipo_escuro} podemos ver uma amostra do protótipo sendo executado em um ambiente com baixa luminosidade. Constatou-se que a luz conseguia se propagar ao longo da fibra, mas que se mostrava com maior intensidade em suas extremidades. A partir disso foi elaborada uma proposta de junção dos LEDs com a fibra ótica que será apresentada mais adiante na  \fullref{sec:malha}.

\begin{figure}[H]
  \begin{center}
    \caption{Protótipo em ambiente com baixa luminosidade}
    \vspace*{0,2cm}
    \includegraphics[width=0.8\textwidth]{./04-figuras/prototipo_escuro}
    \label{fig:prototipo_escuro}
  \end{center}
  \vspace*{-0,9cm}
  \fonte{Elaborada pela autora}\\
\end{figure}

\section{DESCRIÇÃO DA SOLUÇÃO}
\label{sec:solucao}

Para efeitos de organização e com o intuito de facilitar a compreensão, esta seção foi organizada de acordo com os elementos presentes no modelo de instalação interativa proposto por \cite{sogabe2011} já cobertos na \fullref{sec:instalacoes_interativas}. São eles: a interface, representada pelo sensor Microsoft Kinect, que parece invisível dado que o interator precisa apenas estar com o seu corpo presente no espaço para que a obra se materialize; o gerenciamento digital que é feito através de um computador e da utilização de Processing para execução e manutenção do programa; e, por fim, o dispositivo de saída de dados, aqui reprensentado por uma composição entre a malha de LEDs e o Arduino, traduzindo informação em luz.

Devido ao recurso de capturar objetos e movimentos no espaço, optou-se pelo uso do sensor Microsoft Kinect que, conectado a um computador, é responsável por captar a área onde os espectadores se encontram. Integrando-o a uma placa Arduino é possível controlar uma série de LEDs, dispostos em uma grade pendente ao teto. Cada LED possui cabos de fibra ótica \textit{side light} (com emissão de luz lateral) conectados à ele que se iluminam conforme os espectadores caminham sob a grade. Na figura \ref{fig:esquema} podemos ver um esquema de montagem da obra.

\begin{figure}[H]
  \begin{center}
    \caption{Esquema de montagem da obra}
    \vspace*{0,2cm}
    \includegraphics[width=0.8\textwidth]{./04-figuras/esquema}
    \label{fig:esquema}
  \end{center}
  \vspace*{-0,9cm}
  \fonte{Elaborada pela autora}\\
\end{figure}


\subsection{Interface}

O Microsoft Kinect atua como interface da instalação interativa proposta neste trabalho sendo utilizado como fonte de entrada de dados (\textit{input}) para mapear o ambiente tridimensional. Considerando suas limitações, a que mais impactou este projeto é causada pela própria natureza da luz projetada pelo sensor. A luz emitida pelo projetor de infravermelho, ao se deparar com um objeto, gera uma sombra em outro que esteja numa distância maior. Segundo \citeonline{lucero} o resultado é que não se pode determinar a profundidade em zonas afetadas por estas sombras, pois elas criam zonas negras na imagem de profundidade, ou seja, pixels com valor zero, como pode ser visto na figura \ref{fig:kinect_sombras}. O impacto gerado aqui é devido à malha de LEDs se encontrar entre o Kinect e o interator. A malha projeta uma sombra, criando pontos onde a área de intersecção entre o LED e o espectador pode não ser percebida. Para contornar este problema, ao invés de um ponto específico, adotou-se uma região maior que a espessura da sombra para garantir o acendimento dos LEDs.

\begin{figure}[H]
  \begin{center}
    \caption{Efeito das sombras no sensor Kinect}
    \vspace*{0,2cm}
    \includegraphics[width=0.8\textwidth]{./04-figuras/kinect_sombras}
    \label{fig:kinect_sombras}
  \end{center}
  \vspace*{-0,9cm}
  \fonte{Adaptado de \citeonline{lucero}}\\
\end{figure}

Além disso, o Kinect foi configurado através do \textit{script} para considerar a captura de objetos em uma área específica entre a grade e o solo, dessa maneira, apesar de não ser possível impedir a criação de sombras, conforme falado anteriormente, podemos, pelo menos, desconsiderar a grade como uma fonte de entrada de dados. A figura \ref{fig:kinect_exemplo} mostra dois exemplos de imagens criadas a partir dos dados capturados pelo sensor. À esquerda temos um caso sem a configuração mencionada e à direita uma imagem com o sensor já calibrado.

\begin{figure}[H]
  \begin{center}
    \caption{Imagens geradas a partir das informações capturadas pelo sensor Kinect}
    \vspace*{0,2cm}
    \includegraphics[width=0.8\textwidth]{./04-figuras/kinect_exemplo}
    \label{fig:kinect_exemplo}
  \end{center}
  \vspace*{-0,9cm}
  \fonte{Captura de tela gerada pelo sensor Kinect}\\
\end{figure}


Outro ponto importante que podemos observar na figura \ref{fig:kinect_exemplo} é que as imagens possuem apenas \textit{pixels} pretos e brancos. Este comportamento foi programado de maneira intencional. Isto porque, no âmbito desta proposta, nos interessa saber se existe algo (ou alguém) na área em questão e não a distância que possa estar da grade de LEDs. Dessa forma, interatores de diferentes estaturas podem criar o mesmo efeito ao caminhar sob a malha, ainda que, devido à diferença de distância do sensor, pessoas mais baixas pareçam menores na imagem capturada.


\subsection{Gerenciamento digital}

O gerencimento digital da obra é realizado através de um computador que executa um programa escrito em Processing. Interpretando as informações fornecidas pelo Kinect, por meio da biblioteca \textit{Open Kinect for Processing} e, gerando uma imagem bidimensional, que é utilizada por um algorítimo de detecção de cor, é possível determinar quais LEDs devem permanecer apagados e quais devem acender na estrutura. A figura \ref{fig:script} nos dá uma ideia de como o programa foi construído. À esquerda temos uma imagem da câmera com pontos brancos e pretos desenhados sobre ela, onde, cada um deles representa um LED na estrutura física. À direita temos a imagem gerada a partir dos dados do sensor de profundidade, sendo que as áreas brancas identificam a presença do interator. As coordenadas e área dos pontos que visualizamos na imagem à esqueda foram utilizados para capturar fragmentos na imagem à direita. A cor predominante no fragmento de imagem (branco ou preto) determina a informação que é enviada ao Arduino para que este, por fim, acenda ou apague os LEDs. Pontos brancos identificam os LEDs acesos, enquanto os pretos identificam os apagados.

\begin{figure}[H]
  \begin{center}
    \caption{Captura da camera e imagem gerada a partir do sensor de profundidade do Microsoft Kinect associadas à pontos que identificam os LEDs na estrutura física}
    \vspace*{0,2cm}
    \includegraphics[width=0.8\textwidth]{./04-figuras/script}
    \label{fig:script}
  \end{center}
  \vspace*{-0,9cm}
  \fonte{Elaborada pela autora}\\
\end{figure}

\subsection{\textbf{Dispositivo de saída de dados}}

O Arduino e a malha de LEDs, juntos, formam o dispositivo de saída de dados (\textit{output}) que recebe as informações mapeadas pelo sensor Kinect (entrada ou \textit{input}). Cada LED precisa ser controlado individualmente, por isso optou-se pela utilização do Arduino Mega 2560 que possui 54 entradas/saídas digitais, suficientes para atender a proposta apresentada sem adicionar complexidade ao circuito.


Não foi necessário escrever um \textit{software} para executar no Arduino. Utilizou-se a biblioteca Firmata para servir como ponte de comunicação entre o dispositivo e o computador. Através de um programa fornecido pela mesma e enviado para o microcontrolador foi possível gerenciar, a partir da Processing, todos os recursos disponíveis na placa.

\subsubsection{O circuito}

As pernas negativas dos LEDs foram soldadas em um fio conectado na porta \textit{ground} do Arduino, enquanto as pernas positivas foram conectadas à fios e plugadas individualmente em portas digitais. Na figura \ref{fig:breadboard} podemos ver o esquema equivalente a uma linha da matriz de LEDs. As demais são conectadas da mesma maneira e só não foram adicionadas para não poluir a imagem.

\begin{figure}[H]
  \begin{center}
    \caption{Circuito equivalente a uma linha da matriz de LEDs}
    \vspace*{0,2cm}
    \includegraphics[width=0.8\textwidth]{./04-figuras/breadboard}
    \label{fig:breadboard}
  \end{center}
  \vspace*{-0,9cm}
  \fonte{Elaborada pela autora}\\
\end{figure}

Constatou-se a necessidade de isolamento elétrico e de construir \textit{jumpers} plugáveis na placa. Para isso, alguns materiais foram necessários. Além do próprio LED, pinos metálicos foram utilizados para conectá-lo ao fio e este ao Arduino, \textit{cases} e tubos termoretráteis para isolamento das conexões, bem como a fibra ótica para causar o efeito proposto. O quadro  \ref{quadro:materiais} mostra as quantidades e especificações destes materiais por elemento luminescente.

\begin{quadro}[H]
\caption{\label{quadro:materiais}Materiais utilizados por elemento luminescente}
\begin{center}  
  \begin{tabulary}{1\textwidth}{|L|L|L|}
  
  \hline
  \textbf{Quantidade} & \textbf{Material} & \textbf{Especificação} \\ \hline
  01 peça & LED alto brilho azul & 5 mm \\ \hline
  03 peças & tubo termo retrátil & 1 mm de espessura e 5 cm de comprimento \\ \hline
  01 peça & pino metálico & macho \\ \hline
  01 peça & \textit{case} plástico & -  \\ \hline
  02 peças & pino metálico & fêmea \\ \hline
  40 cm & fibra ótica \textit{side light} & 0.5 mm de espessura em pedaços de aproximadamente 3 cm \\ \hline
  40 cm & fibra ótica \textit{side light} & 0.75 mm de espessura em pedaços de aproximadamente 3 cm \\ \hline
  20 cm & fibra ótica \textit{side light} & 1 mm de espessura em pedaços de aproximadamente 3 cm  \\ \hline
  Tamanho variado & fio & AWG 30 \\ \hline
  \end{tabulary}
\end{center}
\vspace*{-0,7cm}
\fonte{Elaborado pela autora}\\
\end{quadro}


\subsubsection{Malha de LEDs}
\label{sec:malha}

A malha é composta primariamente por uma grade de 120 x 80 centímetros que delimita o espaço da instalação onde o espectador pode interagir com a obra. Como podemos ver na figura \ref{fig:malha}, esta grade possui intersecções a cada 20 centímetros, formando uma matriz de 5 linhas por 7 colunas. Cada intersecção possui um LED preso à ela, somando um total de 35 LEDs dispostos na obra.

\begin{figure}[H]
  \begin{center}
    \caption{Esquema da malha de LEDs}
    \vspace*{0,2cm}
    \includegraphics[width=0.8\textwidth]{./04-figuras/malha}
    \label{fig:malha}
  \end{center}
  \vspace*{-0,9cm}
  \fonte{Elaborada pela autora}\\
\end{figure}


Cada LED ocupa uma área na imagem processada pelo sistema computacional e conta com fios de fibra ótica, com emissão de luz lateral, de várias espessuras colados em sua extremidade. Quando o LED acende, a partir da interação com o usuário, a luz é transmitida pela fibra ótica, gerando vários pontos iluminados. Na figura \ref{fig:led_fibra_otica} podemos ver o exemplo de um desses LEDs, sendo que, no primeiro quadro o LED é exibido apagado, no segundo aceso em ambiente com alta luminosidade e, por fim, no terceiro quadro, aceso em ambiente com baixa luminosidade. 

\begin{figure}[H]
  \begin{center}
    \caption{LED com fibra ótica em diferentes condições de luminosidade}
    \vspace*{0,2cm}
    \includegraphics[width=0.8\textwidth]{./04-figuras/led_fibra_otica}
    \label{fig:led_fibra_otica}
  \end{center}
  \vspace*{-0,9cm}
  \fonte{Elaborada pela autora}\\
\end{figure}

A partir da figura \ref{fig:led_fibra_otica} é possível constatar a diferença de exposição em ambientes com distintos graus de luminosidade. Ainda que a luz seja visível ao se observar o LED em um ambiente com alta incidência de luz e, isso viabilize a exposição do trabalho mesmo em condições como esta, se nota uma discrepância considerável quando voltamos nossos olhos para o terceiro quadrante, onde o LED é apresentado em ambiente com baixa luminosidade. A peça ganha destaque e a cor azul se propaga com maior intensidade.

\section{PROJETOS DE INSTALAÇÃO}

A obra foi disposta em um ambiente de laboratório para que fossem realizados testes de funcionamento e iluminação. Na figura \ref{fig:trabalho} podemos ver um exemplo de interação com o público em um ambiente com luz controlada. 
   
   
\begin{figure}[H]
  \begin{center}
    \caption{Interação com o público}
    \vspace*{0,2cm}
    \includegraphics[width=0.8\textwidth]{./04-figuras/trabalho}
    \label{fig:trabalho}
  \end{center}
  \vspace*{-0,9cm}
  \fonte{Elaborada pela autora}\\
\end{figure}


\begin{figure}[H]
  \begin{center}
    \caption{Espaço da instalação na pinacoteca Barão de Santo Ângelo}
    \vspace*{0,2cm}
    \includegraphics[width=0.8\textwidth]{./04-figuras/pinacoteca}
    \label{fig:pinacoteca}
  \end{center}
  \vspace*{-0,9cm}
  \fonte{Adaptado de https://www.ufrgs.br/institutodeartes}\\
\end{figure}

