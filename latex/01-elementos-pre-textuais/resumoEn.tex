
\begin{ABSTRACT}
	\begin{SingleSpace}
	
		\hspace{-1.3 cm}
		This project reports the building process of an interactive computer work of art which uses light as its main source of aesthetical constitution. It is based on a grid of LEDs which light up as the spectator walks below them. Its main components are the Microsoft Kinect, which detects the presence of the spectator, an Arduino board, which controls the LEDs grid, and a computer with Processing software installed, which orchestrates the devices functioning. The concept is an invitation for the user to quit their position of contemplation and identify themselves as a co-author, exploring new ways of experiencing the piece. From a theoretical point of view, this research comprehends topics such as interactivity, computer art, the link between piece and spectator's body and the application of light as material. Besides, it presents reference artists like Jim Campbell, Takahito Matsuo, among others.
		

		\vspace*{0.5cm}\hspace{-1.3 cm}\textbf{Keywords}: Art and Technology. Interactive art. Computer art. Light.
		
		
	\end{SingleSpace}

\end{ABSTRACT}
