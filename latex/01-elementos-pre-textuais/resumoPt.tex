\begin{RESUMO}
\thispagestyle{empty}
	\begin{SingleSpace}
	
		\hspace{-1.3 cm}O presente trabalho relata o processo de construção de uma obra de arte computacional interativa que tem a luz como fonte principal de sua constituição estética. Se trata de uma malha de LEDs que acendem quando o espectador caminha sob eles. Seus principais componentes são o Microsoft Kinect, responsável por detectar a presença do espectador, e um Arduino que, por sua vez, controla a malha de LEDs.  A proposta é um convite para que o fuidor saia de um lugar de contemplação e se descubra como co-autor, explorando novas maneiras de experimentar a obra. 
		 
		
		\vspace*{0.5cm}\hspace{-1.3 cm}\textbf{Palavras-chave}: Tecnologia. Arte computacional. Arte interativa. Microsoft Kinect. Arduino. LED. Luz. Fibra ótica.
		
	\end{SingleSpace}
\end{RESUMO}


