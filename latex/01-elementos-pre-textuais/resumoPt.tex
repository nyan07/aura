\begin{RESUMO}
\thispagestyle{empty}
	\begin{SingleSpace}
	
		\hspace{-1.3 cm}O presente trabalho relata o processo de construção de uma obra de arte computacional interativa que tem a luz como fonte principal de sua constituição estética. Trata-se de uma malha de LEDs que acendem quando o espectador caminha sob eles. Seus principais componentes são o Microsoft Kinect, responsável por detectar a presença do espectador, uma placa Arduino, que controla a malha de LEDs, e um computador com o software Processing, responsável por orquestrar o funcionamento dos dispositivos. A proposta é um convite para que o fruidor saia de um lugar de contemplação e se descubra como co-autor, explorando novas maneiras de experimentar a obra.
		Do ponto de vista teórico a pesquisa engloba questões como interatividade, arte computacional, a relação da obra com o corpo do espectador e a utilização da luz como material. Além disso, apresenta artistas referenciais como Jim Campbell, Takahito Matsuo, entre outros.
		
		\vspace*{0.5cm}\hspace{-1.3 cm}\textbf{Palavras-chave}: Arte e Tecnologia. Arte interativa. Arte computacional. Luz.
		
	\end{SingleSpace}
\end{RESUMO}


